\documentclass{article}
\usepackage{fullpage}
\usepackage[utf8]{inputenc}
\usepackage{indentfirst}
\usepackage{enumerate}
\usepackage{tasks}
\usepackage{exsheets}
\SetupExSheets[question]{type=exam}
\usepackage{amsmath}
\usepackage[portuguese]{babel}
\usepackage{hyperref}



\title{
    \textbf{Primeiro Trabalho - Relatório Parcial} \\
     { \large \textbf{Computação Concorrente (MAB-117) -- 2021/1 REMOTO}} \\ 
   \textit{\textbf{Conversão de Imagens em Escala de Cinza}}
    }
\author{\textbf{Carlos Bravo} \\ \textbf{Markson Arguello} \\ \\ 119136241 \\  119132001}
\date{Agosto 2021}

\begin{document}

\maketitle

\section{Descrição do problema}
    O problema escolhido se trata de converter um diretório com várias fotos em RGB para tons de cinza. O programa deve receber fotos coloridas de entrada e criar novas fotos convertendo as cores em tons de cinza. 
    
    Para que isso aconteça podemos representar a foto como um vetor onde os índices \(3i, 3i + 1 \text{ e } 3i + 2\) representam a quantidade de vermelho, verde e azul do pixel \(i\). Com isso, basta mudarmos os valores adicionando pesos nessas cores para criar a foto em tons de cinza (30\% vermelho, 59\% verde e 11\% azul) \cite{wikiGray}. 
    
    Como vamos converter várias fotos e precisaremos percorrer todo o vetor de cada uma das fotos para isso, uma vez que as fotos são independentes vemos que podemos utilizar a computação concorrente para obter um ganho de desempenho.
    
    
\section{Projeto inicial da solução concorrente}
    Para utilizar a concorrência poderemos utilizar alguma das seguintes abordagens:
     \begin{itemize}
        \item Cada thread computa uma linha ou coluna da foto
        \item As threads se intercalam entre pixels
        \item Cada thread computa uma foto
    \end{itemize}
    Nós escolhemos utilizar a 3º opção pois ela nos permite que cada thread seja responsável por ler, processar e escrever independente a foto, impedindo problemas de concorrência na leitura e escrita de arquivos. Além disso, podemos ter duas abordagens para essa opção:
    \begin{itemize}
        \item As threads lêem os arquivos alternadamente
        \item Cada thread lê o arquivo que está livre no momento
    \end{itemize}
    A escolha da sub-opção será feita de acordo com o tempo de execução, então será feita das duas formas e assim comparadas, para saber qual é melhor para o problema.
    
    As estruturas de dados serão:
    \begin{itemize}
        \item Atributos do arquivo (Globais)
        \item Dados da imagem (Local para a thread responsável)
    \end{itemize}
    
    
    Como entrada é passado um identificador para cada thread, como saída nada é esperado, pois cada thread cuidará de todos as etapas no processo de conversão. 
     
\section{Projeto inicial dos casos de teste}
    Para o teste de corretude usaremos pastas com tamanhos e quantidades diferentes de fotos. Depois será feito um algoritmo sequencial que rodará para as mesmas imagens, e assim basta comparar os pixels de cada uma das imagens.
    
    Para o teste de desempenho será criada uma pasta com uma grande quantidade de fotos, como por exemplo todos os frames da música Never Gonna Give You Up (Rick Astley) \cite{rickrolled}.
    
\begin{thebibliography}{2}

\bibitem{wikiGray}
Nível de cinza, from Wikipedia. Disponível em \url{https://pt.wikipedia.org/wiki/N\%C3\%ADvel_de_cinza}. Acesso em: 02 ago. 2021

\bibitem{rickrolled}
Rick Astley - Never Gonna Give You Up (Official Music Video). Disponível em \url{https://www.youtube.com/watch?v=dQw4w9WgXcQ}. Acesso em: 04 ago. 2021

\end{thebibliography}

\end{document}
